\documentclass[openany,10pt,UTF8]{ctexbook}
\usepackage[a4paper,twoside,width=15cm]{geometry}
\usepackage{amsmath}                                    % 排版数学公式
\usepackage{latexsym}
\usepackage{amsfonts}                                   %数学符号字体库宏包套件,它包含有:amsfonts、amssymb、eufrak 和 eucal 四个宏包。
\usepackage{amssymb}                                    % 定义AMS的数学符号命令
\usepackage{mathrsfs}                                   % 数学RSFS书写字体
\usepackage{bm}                                         % 数学黑体
\usepackage{graphicx}                                   % 支持插图,图形宏包graphics的扩展宏包
\usepackage{color,xcolor}                               % 支持彩色
\usepackage{amscd}
\usepackage[linesnumbered,ruled,vlined]{algorithm2e}
\usepackage{diagbox}
\usepackage{minted}
\usepackage{titlesec}                                   %设置章节格式
\usepackage{enumerate}                                 	%更改enumerate环境格式
\usepackage{hyperref}
\usepackage{subcaption}
\usepackage{minipage-marginpar}
\usepackage{float}%提供float浮动环境
\usepackage{booktabs}%提供命令\toprule、\midrule、\bottomrule
\usepackage{listings}
\usepackage{xcolor}

% 自定义命令,用于角标引用文献和交叉引用
\newcommand{\scite}[1]{\textsuperscript{\cite{#1}}}
\newcommand{\sref}[1]{\textsuperscript{\ref{#1}}}
\newcommand{\bs}[1]{\boldsymbol{#1}}
\newcommand{\romannumber}[1]{\uppercase\expandafter{\romannumeral#1}}

%%对一些autoref的中文引用名作修改
\def\equationautorefname{式}
\def\footnoteautorefname{脚注}
\def\itemautorefname{项}
\def\figureautorefname{图}
\def\tableautorefname{表}
\def\partautorefname{篇}
\def\appendixautorefname{附录}
\def\chapterautorefname{章}
\def\sectionautorefname{节}
\def\subsectionautorefname{小小节}
\def\subsubsectionautorefname{subsubsection}
\def\paragraphautorefname{段落}
\def\subparagraphautorefname{子段落}
\def\FancyVerbLineautorefname{行}
\def\theoremautorefname{定理}
\usepackage{subcaption}
\usepackage{minipage-marginpar}
\usepackage{float}%提供float浮动环境
\usepackage{booktabs}%提供命令\toprule、\midrule、\bottomrule
\usepackage{listings}
\usepackage{xcolor}

% 自定义命令,用于角标引用文献和交叉引用
\newcommand{\scite}[1]{\textsuperscript{\cite{#1}}}
\newcommand{\sref}[1]{\textsuperscript{\ref{#1}}}
\newcommand{\bs}[1]{\boldsymbol{#1}}
\newcommand{\romannumber}[1]{\uppercase\expandafter{\romannumeral#1}}

%%对一些autoref的中文引用名作修改
\def\equationautorefname{式}
\def\footnoteautorefname{脚注}
\def\itemautorefname{项}
\def\figureautorefname{图}
\def\tableautorefname{表}
\def\partautorefname{篇}
\def\appendixautorefname{附录}
\def\chapterautorefname{章}
\def\sectionautorefname{节}
\def\subsectionautorefname{小小节}
\def\subsubsectionautorefname{subsubsection}
\def\paragraphautorefname{段落}
\def\subparagraphautorefname{子段落}
\def\FancyVerbLineautorefname{行}
\def\theoremautorefname{定理}

\begin{document}
\frontmatter
\begin{titlepage}
	\centering % Center everything on the title page
	\scshape % Use small caps for all text on the title page
	\vspace*{1.5\baselineskip} % White space at the top of the page
% ===================
%	Title Section 	
% ===================

	\rule{13cm}{1.6pt}\vspace*{-\baselineskip}\vspace*{2pt} % Thick horizontal rule
	\rule{13cm}{0.4pt} % Thin horizontal rule
	
		\vspace{0.75\baselineskip} % Whitespace above the title
% ========== Title ===============	
	{	
        \Huge 
			《空天飞行力学》课程笔记 \\	
    }
% ======================================
		\vspace{0.75\baselineskip} % Whitespace below the title
	\rule{13cm}{0.4pt}\vspace*{-\baselineskip}\vspace{3.2pt} % Thin horizontal rule
	\rule{13cm}{1.6pt} % Thick horizontal rule
	
		\vspace{1.75\baselineskip} % Whitespace after the title block
% =================
%	Information	
% =================
	{\large 作者:欧阳嘉鸿\\
		\vspace*{1.2\baselineskip}
	ouyangjiahong22@nudt.edu.cn} \\
	\vfill
如果你有任何问题或评论, \\ \vspace{1mm} 可以通过邮件 \url{ouyangjiahong22@nudt.edu.cn} 联系我\\ \vspace{1mm}

\end{titlepage}
\tableofcontents % 生成目录
\mainmatter
%%% 请在下方输入正文 %%%
\part{弹道学部分}
\chapter{理论基础}	
\input{坐标系统}
\input{坐标系间的方向余弦阵及矢量导数的关系}
\input{变质量力学基本原理}
\input{常用坐标系统及其相互转换}
\input{时间系统}
\input{远程火箭控制系统}
\input{空天飞行器所受的力和力矩}
\chapter{运动方程及弹道方程}
\input{空天飞行器的一般运动方程及计算方程}
\input{地面发射坐标系中的空间弹道计算方程}
\input{速度坐标系内建立的空间弹道方程}
\chapter{上升段运动特性与轨迹设计}
\input{用于方案论证阶段简化的纵向方程}
\input{上升段运动特性分析}
\input{主动段终点速度、位置及全射程估算}
\chapter{远程火箭主动段运动特性分析}
\input{火箭设计参数的选择}
\input{多级火箭}
\input{主动段飞行程序的选择}

\part{轨道学部分}
\chapter{在轨段运动特性与轨迹设计}
\input{在轨段运动特性}
\input{在轨段轨迹设计}
\chapter{再入段运动方程与运动特性}
\input{再入段运动特性}
%%% 正文结束 %%%
\end{document}